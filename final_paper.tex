\documentclass{article}
\usepackage[utf8]{inputenc}
\usepackage{multicol}

\title{gofish}
\author{Aubrey Spannagel, Jordan Hiatt, Alyssa Felzien }
\date{March 2021}

\begin{document}
\maketitle
\begin{multicols}{2}
[
\section{Introduction}
]
Our project was to create a Go Fish game using the programming language Solidity, the Remix IDE, and a GUI made with HTML and CSS. We decided to make Go Fish because it had simple game logic and would allow us to learn more about Solidity and Remix. This paper will describe what Solidity and Remix are, how we used them in our project, unique problems with smart contracts, and our final results. 
\end{multicols}

\end{multicols}
\begin{multicols}{2}
[
\section{Remix and Solidity}
]

Remix is an open source web and desktop IDE that has access to many plugins and allows for fast development. It is used to for smart contract development and allows users to compile and deploy their contracts into 3 different environments, JavaScript VM, Injected Provider, and Web3 Provder. For our project we used the browser version and deployed to the JavaScript VM environment for testing and to the Web3 Provider for our GUI.

Solidity is a an object-oriented language that is statically typed and supports inheritance, libraries, and complex user-defined types. It is used to implement smart contracts. Smart contracts are similar to microservices on the web, and anyone can use them. These programs 

\end{multicols}

\begin{multicols}{2}
[
\section{DeckOfCards} (Jordan)
]
1. Go into set up of all of that
2. Concept of gas, memory and storage
\end{multicols}

\begin{multicols}{2}
[
\section{GoFishGame}
]
In the GoFishGame project we created two smart contracts, DeckOfCards.sol and GoFishGame.sol. DeckOfCards.sol provided us with the methods and variables to easily interact with a DeckOfCards inside GoFishGame. In Solidity, you can use a smart contract within another smart contract by using the import command. For example, \begin{verbatim} import "./DeckOfCards.sol"; \end{verbatim} Like in other object oriented programming languages, this makes it easier to organize code and build complex systems.
In addition to using smart contracts, we also needed to have random numbers in our project to shuffle the deck and to allow the computer to play. Unfortunately, Solidity can not create random numbers. Calculations for pseudo random numbers cost too much so basic solutions are used.  

\end{multicols}
\begin{multicols}{2} [
\section{Attaching to UI} (Aubrey)
]
In order to connect the Smart Contract to the Ethereum IDE, we had to start with an HTML and CSS base. Ethereum includes a Web3.js extention that allows the connection between the IDE and a local node.To do this, we had to install geth, a program that allows an Ethereum node to be ran locally. After creating and running the node, we needed to add the contract credentials to the HTML file. Thankfully, Remix was helpful enough to supply the JavaScipt code with the credentials already filled in; this made connecting to the Blockchain simple.  
\end{multicols}

\begin{multicols}{2} [
\section{Conclusion}
]
1. Final thoughts, end results, 
\end{multicols}


\section{References} 
https://www.sitepoint.com/solidity-pitfalls-random-number-generation-for-ethereum 

(format later-Aubrey)
https://www.youtube.com/watch?v=hcTPjpPvas8 
https://livecodestream.dev/post/interacting-with-smart-contracts-from-web-apps/
https://metamask.zendesk.com/hc/en-us/articles/360028141672-How-to-deposit-receive-tokens-to-your-MetaMask-Wallet

https://ethereum.org/en/developers/docs/smart-contracts/deploying/ 

https://docs.soliditylang.org/en/v0.8.2/
https://remix-ide.readthedocs.io/en/latest/
\end{document}
